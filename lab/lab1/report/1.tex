\documentclass[12pt,a4paper]{ctexart}
\usepackage{listings}
\usepackage{color}
\usepackage{amsmath}

\definecolor{dkgreen}{rgb}{0,0.6,0}
\definecolor{gray}{rgb}{0.5,0.5,0.5}
\definecolor{mauve}{rgb}{0.58,0,0.82}

\lstset{ %
  language=Octave,                % the language of the code
  basicstyle=\footnotesize,           % the size of the fonts that are used for the code
  numbers=left,                   % where to put the line-numbers
  numberstyle=\tiny\color{gray},  % the style that is used for the line-numbers
  stepnumber=2,                   % the step between two line-numbers. If it's 1, each line
                                  % will be numbered
  numbersep=5pt,                  % how far the line-numbers are from the code
  backgroundcolor=\color{white},      % choose the background color. You must add \usepackage{color}
  showspaces=false,               % show spaces adding particular underscores
  showstringspaces=false,         % underline spaces within strings
  showtabs=false,                 % show tabs within strings adding particular underscores
  frame=single,                   % adds a frame around the code
  rulecolor=\color{black},        % if not set, the frame-color may be changed on line-breaks within not-black text (e.g. commens (green here))
  tabsize=2,                      % sets default tabsize to 2 spaces
  captionpos=b,                   % sets the caption-position to bottom
  breaklines=true,                % sets automatic line breaking
  breakatwhitespace=false,        % sets if automatic breaks should only happen at whitespace
  title=\lstname,                   % show the filename of files included with \lstinputlisting;
                                  % also try caption instead of title
  keywordstyle=\color{blue},          % keyword style
  commentstyle=\color{dkgreen},       % comment style
  stringstyle=\color{mauve},         % string literal style
  escapeinside={\%*}{*)},            % if you want to add LaTeX within your code
  morekeywords={*,...}               % if you want to add more keywords to the set
}

\begin{document}
\section{实验目的与内容}
\subsection{实验目的}
通过本次实验熟悉RISC指令集(RV32I指令集)的部分指令和编码格式
\subsection{实验内容}
\subsubsection{斐波那契数列}
编写汇编程序,计算斐波那契数列的第$N$项($0 \leq N \leq 30 $
)。初始时,$N$ 的值保存在 $R_{2}$
 中。程序执行完成后,数列的第 $N$
 项保存在 $R_{3}$ 中。
\subsubsection{大整数处理}
编写汇编程序,计算斐波那契数列的第 $N$项($0 \leq N \leq 80 $
)。初始时,$N$的值保存在  中。程序执行完成后,数列的第 $N$
 项保存在 $R_{3}$ 和 $R_{4}$ 中,其中 $R_{3}$
 存储结果的高 32 位,$R_{4}$ 存储结果的低 32 位。
\subsubsection{导出COE文件}
完成汇编程序的编写之后,导出指令段的 COE 文件,以供后续实验使用。

\section{逻辑设计}
\subsection{斐波那契数列}
斐波那契数列计算公式为
$$
F_N=
\begin{cases}
    1&1\leq N \leq 2\\
    F_{N-1}+F_{N-2}&N \textgreater 3
\end{cases}
$$
其中,$N$为整数。
\begin{lstlisting}[caption={斐波那契数列}]
.data
    #设置N的值
    N:
     .word 2
.text
MAIN:
    #x2用于存储N的值(即R2),x1存储结果(即R3)
    lw x2,N
    #先判断N是否大于2,若否,则直接将结果设置为1并跳转到代码结尾
    li t1,3
    bge x2,t1,N3
    addi x1,x1,1
    bge x0,x0,END
N3:
    addi x2,x2,-2
    li t1,1
    li t2,1
    #计算斐波那契数列,直到x2等于0时退出循环,结果在x1中
FOR:
    add x1,t1,t2
    addi t1,t2,0
    addi t2,x1,0
    addi x2,x2,-1
    blt x0,x2,FOR
END:
\end{lstlisting}
\subsection{大整数处理}
代码整体设计思路同上,但是由于斐波那契数列第80项已经大于$2^{32}$,
故需要两个寄存器来进行计算,分别存储结果的低32位和高位。\par
在进行低32位的加法计算时需要判断是否进位,判断方法为将加法结果与
任意一个加数比较,若结果小于加数,则产生了进位。在进行高位加法计算时
需要加上进位。\par
代码如下:
\begin{lstlisting}[caption={大整数处理}]
.data
    N:
     .word 69
.text
MAIN:
    #x2用于存储N的值(即R2),x3存储结果高位(即R3),x4存储结果低位(即R4)
    lw x2,N
    li t1,3
    bge x2,t1,N3
    addi x4,x4,1
    bge x0,x0,END
N3:
    addi x2,x2,-2
    #t1为高位,t2为低位
    li t1,0
    li t2,1
    #t3为高位,t4为低位
    li t3,0
    li t4,1
FOR:
    #低位相加
    add x4,t2,t4
    #t5存储进位
    sltu t5,x4,t2
    #高位相加
    add x3,t1,t3
    #高位加进位
    add x3,x3,t5
    addi t1,t3,0
    addi t2,t4,0
    addi t3,x3,0
    addi t4,x4,0
    addi x2,x2,-1
    blt x0,x2,FOR
END:
\end{lstlisting}
\subsection{导出COE文件}
将代码段与数据段分别导出,并在文件开头加上如下代码:
\\
$
memory \mbox{\underline{\space}} initialization\mbox{\underline{\space}}radix  = 16;
$\\
$
memory\mbox{\underline{\space}}initialization\mbox{\underline{\space}}vector =
$\par
结果如下:
\begin{lstlisting}[caption={斐波那契数列数据段}]
memory_initialization_radix  = 16;
memory_initialization_vector =
00000006
00000000
\end{lstlisting}
\begin{lstlisting}[caption={斐波那契数列代码段}]
memory_initialization_radix  = 16;
memory_initialization_vector =
0fc10117
00012103
00300313
00615663
00108093
02005263
ffe10113
00100313
00100393
007300b3
00038313
00008393
fff10113
fe2048e3
\end{lstlisting}
\begin{lstlisting}[caption={大整数处理数据段}]
memory_initialization_radix  = 16;
memory_initialization_vector =
00000050
00000000
\end{lstlisting}
\begin{lstlisting}[caption={大整数处理代码段}]
memory_initialization_radix  = 16;
memory_initialization_vector =
0fc10117
00012103
00300313
00615663
00108093
04005063
ffe10113
00000313
00100393
00000e13
00100e93
01d38233
00723f33
01c301b3
01e181b3
000e0313
000e8393
00018e13
00020e93
fff10113
fc204ee3
\end{lstlisting}
\section{总结}
通过本次实验初步掌握了RV32I指令集部分命令的使用以及编译和运行的方法。
\end{document}
