\documentclass[12pt,a4paper]{ctexart}
\usepackage{graphicx}
\usepackage{wrapfig}
\usepackage{float}
\usepackage{siunitx}
\usepackage{subfigure}
\usepackage{caption}
\usepackage{natbib}
\usepackage{listings} % 引入listings宏包用于插入代码
\usepackage{xcolor} % 引入xcolor宏包以支持更多的颜色设置

% 设置Verilog代码样式
\lstdefinestyle{verilog}{
    language=Verilog, % 设置语言为Verilog
    basicstyle=\small\ttfamily, % 设置基本字体样式
    keywordstyle=\color{blue}, % 关键字颜色设置
    commentstyle=\color{gray}\ttfamily, % 注释颜色和样式设置
    stringstyle=\color{red!60!black},
    numbers=left, % 行号在左边显示
    numberstyle=\tiny,
    frame=single, % 添加单线框
    rulecolor=\color{black!30}, % 边框颜色
    breaklines=true, % 允许自动换行
}

\title{Homework2}
\author{张子康 \ PB22020660}
\date{2024年04月06日}

\begin{document}
\maketitle
\newpage
\section*{1}
\subsection*{1.1}
定义x1=es,x2=eax,x3=ecx,x4为计算后的地址
\begin{lstlisting}
slli x3,3
add x4,x2,x3
addi x4,x4,0x5678
lw x1,x4
addi x1,x1,0x5678
sw x1,x4
\end{lstlisting}
\subsection*{1.2}
x1,x2为eax,x3,x4为ecx\\
x10,x11存储$eax+ecx\times8+0x11223344$结果\\
x12,x13存储读取的结果与计算结果\\
\begin{lstlisting}
//计算ecx*8
addi x5,x3,0
srli x5,5
slli x4,3
slli x3,3
add x3,x3,x5
//eax+ecx*8
add x11,x2,x4
sltu x6,x11,x2
add x10,x1,x2
add x10,x10,x6
//eax+ecx*8+0x11223344
addi x11,x11,0x11223344
sltiu x6,x11,0x11223344
add x10,x10,x6
//读取寄存器
lw x12,x10
lw x13,x11
addi x13,x13,0x12345678
sltiu x6,x13,0x12345678
add x12,x12,x6
//写寄存器
sw x12,x10
sw x13,x11
\end{lstlisting}
\section*{2}
\subsection*{2.1}
0x56
\subsection*{2.2}
0x34
\section*{3}
\subsection*{3.1}
$-2147483648\leq t_{1}\leq-2147475420$
\subsection*{3.2}
jal可以到达$pc+1048574\mbox{或}pc-1048576$。
blt可以到达$pc+4094\mbox{或}pc-4096$。
\section*{4}
\subsection*{4.1}
\subsubsection*{4.1.1}
在基于栈的内存管理中,\$sp寄存器用于指示栈顶位置,负责管理和维护栈内存
的分配与回收,而 \$fp
寄存器则用于记录当前函数调用帧
的底部地址,以便于访问局部变量和函数参数等栈帧内的
数据。
\subsubsection*{4.1.2}
指针
\subsubsection*{4.1.3}
\$$ra$占$32$字节,\$$fp$占$32$字节,
指针占$16$字节,alloca的空间占$16$字节,
第二个变量占$16$字节,一共$14Byte$
\subsubsection*{4.1.4}
\begin{lstlisting}
b \$ra    
\end{lstlisting}
\subsubsection*{4.1.5}
\begin{lstlisting}
b \$ra    
\end{lstlisting}
\subsection*{4.2}
\subsubsection*{4.2.1}
$\$fp=0x24000000,\$sp=23FFFFB0$
\subsubsection*{4.2.3}
\begin{lstlisting}
jirl \$ra,main,0    
\end{lstlisting}
\subsubsection*{4.2.4}
\begin{lstlisting}
lu12i.w t0,0x23fff
addi.w t0,t0,0xffc
ld.w a1,t0,0
jirl \$ra,main,0    
\end{lstlisting}
\end{document}