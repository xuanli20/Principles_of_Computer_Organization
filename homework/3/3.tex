\documentclass[12pt,a4paper]{ctexart}
\usepackage{graphicx}
\usepackage{wrapfig}
\usepackage{float}
\usepackage{siunitx}
\usepackage{subfigure}
\usepackage{caption}
\usepackage{natbib}
\usepackage{listings} % 引入listings宏包用于插入代码
\usepackage{xcolor} % 引入xcolor宏包以支持更多的颜色设置

% 设置Verilog代码样式
\lstdefinestyle{verilog}{
    language=Verilog, % 设置语言为Verilog
    basicstyle=\small\ttfamily, % 设置基本字体样式
    keywordstyle=\color{blue}, % 关键字颜色设置
    commentstyle=\color{gray}\ttfamily, % 注释颜色和样式设置
    stringstyle=\color{red!60!black},
    numbers=left, % 行号在左边显示
    numberstyle=\tiny,
    frame=single, % 添加单线框
    rulecolor=\color{black!30}, % 边框颜色
    breaklines=true, % 允许自动换行
}

\title{Homework3}
\author{张子康 \ PB22020660}
\date{2024年04月15日}

\begin{document}
\maketitle
\newpage
\section{}
\subsection{}
单精度浮点数:符号位:1位,
阶码:8位,
尾数:22位。\par
双精度浮点数:
符号位:1位,
阶码:11位,
尾数:51位。
\subsection{}
单精度浮点数:
0.2:cdcc4c3e,
0.7:3333333f,
0.9:6666663f,0.2 + 0.7 的结果:6666663f。\par
双精度浮点数:0.2:9a9999999999c93f,
0.7:666666666666e63f,
0.9:cdccccccccccec3f,
0.2 + 0.7 的结果:ccccccccccccec3f。\par
在计算机中小数无法精确表示,0.2,0.7和0.9
在单精度下恰好0.2+0.7=0.9,而在双精度下不满足。

\section{}
\subsection{}
\subsubsection{}
\begin{enumerate}
    \item PC元件:给出指令所在的地址;
    \item 指令存储器:根据PC的值取出指令;
    \item 控制器:生成控制信号,包括ALU的opcode,寄存器写使能信号,控制输入ALU单元的数据,控制写回寄存器的数据来源;
    \item 寄存器:根据给出的地址取出操作数,向制定的地址写入结果;
    \item MUX:控制输入ALU单元的数据时来自寄存器还是立即数,控制写回寄存器的数据来源,控制下一个PC的来源;
    \item ALU:进行加法运算并输出结果;
    \item ALU控制器:产生ALU的opcode;
    \item ImmGen:根据指令对立即数做符号扩展,产生操作数。
\end{enumerate}
\subsubsection{}
\begin{enumerate}
    \item PC元件:给出指令所在的地址;
    \item 指令存储器:根据PC的值取出指令;
    \item 控制器:生成控制信号,包括ALU的opcode,寄存器写使能信号,控制输入ALU单元的数据,控制写回寄存器的数据来源;
    \item 寄存器:根据给出的地址取出操作数,向制定的地址写入结果;
    \item MUX:控制输入ALU单元的数据时来自寄存器还是立即数,控制写回寄存器的数据来源;
    \item ALU:进行比较运算并输出结果;
    \item ImmGen:根据指令对立即数做符号扩展,产生操作数。
    \item Add:计算需要跳转到的pc;
    \item MUX:控制输入ALU单元的数据时来自寄存器还是立即数,控制写回寄存器的数据来源,控制下一个PC的来源;
\end{enumerate}
\subsection{}
\subsubsection{}
pc+4无法写回寄存器
\subsubsection{}
将pc+4的结果连接到控制写回到寄存器数据的MUX上,同时将其改为4选1的数据选择器。
\subsubsection{}
没有改变(应该吧)
\section{}
\subsection{}
\begin{enumerate}
    \item (假设是A-B),对B取反与A相加,再将结果+1;
    \item 设计算结果为C,需要将A右移B位。先计算'd32-B的结果存储在B中(减法计算见上),C左移一位,然后加上A+32'h80000000的进位,将A左移一位,同时B-1。重复上述运算直到B==0。
\end{enumerate}
\subsection{}
设该操作数为src,则操作为src+0
\section{}
\subsection{}
Loongarch32中opcode和function部分是连
在一起的且opcode不是定长(相当于每个opcode对应一个运算),但是在RISCV中
是分开的且opcode定长。\par
在译码器设计时,Loongarch32的每个opcode对应一种运算,在RISCV中除了
要处理opcode,还要根据func段来判断具体运算。
\subsection{}
RISC-V中的rs1,rs2,rd位置固定,可以直接读取,译码和寄存器堆读取不存在依赖关系。但是
在Loongarch32中对于每条指令需要具体判断,译码和寄存器堆读取存在依赖关系,可能会导致延迟增加。
\end{document}